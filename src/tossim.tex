\label{subsec:tossim}
TOSSIM\cite{tossim} is a simulator for TinyOS sensor networks.
All the protocols and systems simulated using TOSSIM are written in nesC \cite{nesC},
an extension to C programming language designed to meet the specification and 
restrictions of TinyOS.
 
TOSSIM provides debugging facilities having several debugging modes 
(such as boot, clock,task,led,...) some of which are being used in TinyOS code
and others are reserved for applications components. Compiling TinyOS for mote
hardware removes the debug statements.
It also provides support for network monitoring and packet injection
through SerialForwarder\footnote{\url{http://docs.tinyos.net/index.php/Mote-PC_
serial_communication_and_SerialForwarder}}, the TinyOS interface tool.

For radio communication, TOSSIM provides two modes: the simple mode when bits
are transmitted without errors, and the lossy mode, when for every pair of communicating
 motes a number between 0 and 1 
denotes the probability with which every received bit will be corrupted. These 
mapping are defined in a configuration file which is given as an argument at boot
time.

TinyViz is an extensible GUI which, besides offering a visual layout of the 
WSN, offers support for debugging and interaction with TOSSIM. Also, traffic can
be seen using TinyViz and extra functionality added through plugins
\footnote{\url{http://www.tinyos.net/tinyos-1.x/doc/tutorial/lesson5.html}}.

